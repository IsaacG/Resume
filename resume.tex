\documentclass{resume}

\usepackage[left=0.75in,top=0.6in,right=0.75in,bottom=0.6in]{geometry}

\usepackage{hyperref}
\usepackage{multicol}
\usepackage{verbatim}

\name{Isaac Good}
\address{ \href{mailto:job@isaacgood.com}{job@isaacgood.com} \hfill San Jose, CA USA \hfill \href{http://www.isaacgood.com/}{www.isaacgood.com} }

\begin{document}

  \begin{rSection}{Skills}
    \begin{tabular}{ @{} >{\bfseries}l @{\hspace{6ex}} l }
      Languages & {\tiny most proficient to familiar}: Python, shell scripting (bash, sh), \\
       & Golang, Perl, C, C++, DB2, MySQL, sqlite, \LaTeX \\
      Technical tools & awk, sed, git, GNU coreutils, *inux, and more \\
      Targeted platforms & UNIX, Linux, Windows, PalmOS, Blackberry, J2ME mobile devices \\
      Personal skills & Strong oral and written communication; \\
       & organized, detail oriented, motivated, fast to learn and adapt
    \end{tabular}
  \end{rSection}

  \begin{rSection}{Experience}
    \begin{rSubsection}{Two Sigma, Inc}{May 2020 - Present}{Reliability Engineer}{New York City, NY}
      \item Improve systems, processes, code, tooling. Build automation. Identify and eliminate pain points. Work with partners to help them benefit from SRE best practices. Collaborate across teams to deliver value across more of the firm.
      \item Work with partner teams to identify areas where data issues could be detected earlier and implement checks to flag issues, along with bringing a third team on board to traige flagged issues.
      \item Work with partner teams to generate data from their systems, then analyzed the data to provide recommendations which can be used to reduce toil and improve availability.
      \item Identify and rectify gaps which cause pain (toil, reduced resiliency, hard to discover information), both to my team and to our customers.
      \item Design, build, deploy tooling used to discover, track, test and lint configuation files.
      \item Implement team best practices across process, documentation, code standards, testing and more.
      \item Work with others to help them learn, understand and implement SRE best practices.
      \item Plan and execute various system and platform upgrades.
      \item Identify projects which have accrued tech debt over time and overhaul, refactor or rewrite tools to simplify them, improve maintainability and the ability to continue growing them, and reduce issues correlated with changes.
      \item Create and deliver classes to peers on various technologies.
    \end{rSubsection}

    \begin{rSubsection}{Google, Inc}{Jan 2019 - May 2020}{Site Reliability Engineer - Google Drive}{Sunnyvale, CA}
      \item Responsible for availability, uptime and health of \href{https://drive.google.com}{Google Drive}
      \item Participate in the on call rotation, owning any production issues or emergencies
    \end{rSubsection}

    \begin{rSubsection}{Google, Inc}{Nov 2016 - Dec 2018}{Site Reliability Engineer - Google Network}{Sunnyvale, CA}
      \item Responsible for availability, uptime and health of Google's internal network
      \item Participate in the on call rotation, owning any production incidents or emergencies
      \item Designed and coded (Python) various components of the network's upgrade automation
      \item Mentored Google SREs, new and old. Gave classes, provided coding guidance, conducted interviews
    \end{rSubsection}

    \begin{comment}
      \begin{rSubsection}{Machine Zone, Inc}{Jul 2016 - Oct 2016}{Senior Site Reliability Engineer}{Palo Alto, CA}
        \item Team lead, responsible for creating and enforcing SRE policies
        \item On call, responding to pages and maintaining production reliability of all of MZ's production systems
        \item Identify pain points and take lead in engineering solutions, removing the pain
        \item Define problems, organizing and assisting others to allow them to create solutions
        \item Change procedures and process based on what isn't working well and working with others to improve the status quo
        \item Work with Puppet, Foreman, Salt, Nginx, ELK and many more technologies
      \end{rSubsection}
    \end{comment}

    \begin{rSubsection}{Google, Inc}{Apr 2013 - Jun 2016}{Site Reliability Engineer - Spanner}{Mountain View, CA}
      \item Responsible for availability, uptime and health of \href{https://en.wikipedia.org/wiki/Spanner_(database)}{Spanner}
      \item Developed and documented best practices for customer data loading
      \item Own the SRE side of Spanner's release deployment and automation
      \item Scoped out, planned and owned supporting and modifying Spanner for complex production changes
      \item Helped deploy various Spanner features and bug fixes
      \item Ditto to Network SRE for oncall, automation, mentoring
    \end{rSubsection}

    \begin{rSubsection}{OANDA Corporation}{Jan 2012 - Jan 2013}{Software Developer}{Toronto, ON}
      \item Created new and maintained old scripts used to report on the state of the system
      \item Collaborated with other teams to improve tools they maintain
      \item Added functionality into large, critical system components (C++)
      \item Worked with project managers and indirectly clients to meet client requirements
    \end{rSubsection}

    \begin{rSubsection}{University of Toronto}{Jul 2009 - Jan 2013}{System Administrator}{Toronto, ON}
      \item Managed users, automation and update tasks, packages, Apache Web Server, DHCP, NFS
      \item Maintain 130 servers, keeping them up to date, dealing with failures, security breaches
      \item Install and update hardware as required.  Set up automated PXE imaging.
      \item Assistant system admin: Jul 2009 - Dec 2011. Part time system admin: Jan 2012 - Jan 2013
    \end{rSubsection}

    \begin{rSubsection}{University of Toronto}{Jul 2009 - Dec 2011}{Graduate Student}{Toronto, ON}
      \item Researched botnet behaviour \textbf{(Jul 2010 - Dec 2011)}
      \item Collected data related to domains and built identifying features
      \item Implemented a classifier to determine if a domain is malicious using WEKA and a learning algorithm
      \vspace{1ex}
      \item Extended Forensix, a kernel module for secure system call logging \textbf{(Jul 2009 - Jul 2010)}
      \item Studied approaches for defining and detecting similar traces of system calls for pruning purposes
    \end{rSubsection}

    \begin{rSubsection}{National Benefit Authority}{Summer 2009 (May - Jul)}{Software Developer}{Toronto, ON}
      \item Added features to an MS Access DB frontend
      \item Normalized the Access DB schema and ported data across schemas
    \end{rSubsection}

    \begin{rSubsection}{OANDA}{Summer 2008 (May - Aug)}{Software Developer}{Toronto, ON}
      \item Created a server application (C++) and protocol plus a web based interface (Perl) to access it
      \item Rewrote scripts to use a standard library, allowing easy and seamless upgrading of many scripts
      \item Wrote applications that reuse existing company libraries, greatly reducing development cycle time
      \begin{comment}
        \item Added features to existing code, extending an existing Perl script 
        \item Learned to write object oriented Perl libraries and gained experience writing very OOP C++ code
      \end{comment}
    \end{rSubsection}

    \begin{rSubsection}{Research In Motion}{Summer 2007 (May - Aug)}{Software Developer}{Toronto, ON}
      \item Designed, prototyped and built a web application to access server admin settings.
      \item Rolled out and replaced an existing, tool with a newly and completely rewritten version.
      \item Documented, reported weekly and supported multiple teams throughout the rollout.

      \begin{comment}
        \item Designed a prototype and implemented a web application to access administrative settings on a server using SOAP-based HTTP communication
        \item Debugged existing code and tested new code, ensuring the code will work in the future and save the company time and money having to deal with broken code later on
        \item Distributed a new version of a tool for use by multiple teams ensuring everyone involved was using and familiarizing themselves with the latest tools
        \item Provided technical support to product’s users, freeing them up from having to spend time figuring out what they had trouble with and allowing them to spend the time productively
        \item Wrote user friendly instructions on use of the new application, allowing the application to enter use quicker than it otherwise would have taken 
        \item Reported weekly on application status to two development groups, helping them keep on schedule
      \end{comment}
    \end{rSubsection}

  \end{rSection}

  \begin{rSection}{Education}
    {\bf Google} \hfill {\em 2013 - 2016} \\
    { Working in a fast paced environment, constantly exposed to new technologies, ideas and architectures. While I have taken a handful of classes on various topics at Google, my day to day exposure to people, projects, problems and solutions has taught me much more about technologies and designing software. }

    {\bf University of Toronto, BASc, Engineering} \hfill {\em 2005 - 2009}

\begin{comment}
    { \small Key courses: Computer Programming, Algorithms, Data Structures and Languages (A+); Communication and Design I \& II (A- \& A+); Operating Systems (A); Algorithms and Data Structures (A+); Introduction to Databases (A); Compilers and Interpreters (A+); Computer Systems Programming (A+); Computer Networks (A); Computer Security (A); Parallel Programming (A); Optimizing Compilers (A-) } \\
    { GPA: 3.08 }
\end{comment}

    {\bf Curiousity, Freenode and the Internet} \hfill {\em 2004 - Present} \\
    { Most of my technical know-how comes from my curiousity to learn how to make things work. With my love for hacking together solutions and my desire to automate everything combined with the amazing online communities (especially Freenode's IRC network and the Archlinux community), I've had unbounded access to knowledge and expert guidance.}

  \end{rSection}

  \begin{rSection}{Independent Projects}
    \begin{rSubsection}{}{}{}{}
      \item \textbf{pacman} Contributed patches to the Archlinux package manager, pacman, making use of the newer features of bash
      \item \textbf{newsbeuter} Added features as well as debugged and patched bugs in the ncurses RSS client, newsbeuter
      \item \textbf{irssi} Wrote several scripts to extend the ncurses IRC client, irssi, and was added as a contributor to the Stalker.pl script
      \item \textbf{More...} of my coding and projects are available via my website, \href{http://www.isaacgood.com/}{isaacgood.com}
    \end{rSubsection}
  \end{rSection}


\begin{comment}
    \begin{rSubsection}{OANDA Corporation}{Jan 2012 - Mar 2012}{Integrator}{Toronto, ON}
      \item Deployed software packages in a testing environment
      \item Detected and stopped bugs before software reached production machines
      \item Learned the software system on a broad, high level
      \item Interacted and collaborated with members of many other teams across
    \end{rSubsection}
\end{comment}

\begin{comment}
    \begin{rSubsection}{Artificial Perception Laboratory (APL)}{May 2007 - Aug 2007}{Team Lead}{Toronto, ON}
      \item Conducted a broad range of projects to help further APL’s research, advancing APLs position in the industry
      \item Led a team of five volunteer students, delivering instructions on a weekly basis and gathering progress reports
      \item Reported weekly to coordinator and other team leaders allowing the different teams to complement each other's productivity
    \end{rSubsection}
\end{comment}

\begin{comment}
    \begin{rSubsection}{Rent Magic}{May 2006 - Aug 2006}{Software Developer}{Toronto, ON}
      \item Designed and implemented a PDA based front end and server backend to interact with a MySQL database producing a valuable product for Rent Magic
      \item Designed API structure, application structure and UI layout, data protocol creating a framework that can be used in the future for many other projects
      \item Kept supervisor up to date on progress and impediments, keeping the project on schedule
      \item Tested application, ensuring maximum customer satisfaction
    \end{rSubsection}
\end{comment}

\begin{comment}
  \begin{rSection}{Academic Projects}
    \begin{rSubsection}{University of Toronto}{2005 - 2010}{Student}{Toronto, ON}
      \item Made HTTP server, UDP based file-transfer client/server, IRC-esque client/server all in C
      \item Created a C-like compiler using bison and flex
      \item Optimized gcc intermediate output files
      \item Implemented malloc/realloc/free in C
      \item Wrote parts of a multithreading OS
    \end{rSubsection}
  \end{rSection}
\end{comment}





\end{document} 
